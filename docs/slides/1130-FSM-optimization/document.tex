\section{Objectives}
\todo{highlight one-hot state vector}
\todo{highlight the timing vs space trade-offs of mealy vs moore FSM in context of verilog}
\todo{Consider describing the stages of Synthesis}
\begin{enumerate}
  \item Analyse and design both Mealy and Moore sequential circuits with multiple inputs and multiple outputs
  \item Convert between Mealy and Moore designs
  % \item Reduce the number of states in a state table using row reduction and
  %   implication tables
  %  \item Perform a state assignment using the guideline method
  %  \item Partition a system into multiple state machines
  \item Perform a state assignment using the guideline method
   \item Reduce the number of states in a state table using row reduction and
     implication tables
   %\item Partition a system into multiple state machines
\end{enumerate}

\section{Mealy vs Moore Finite State Machines}
\begin{definition}[Finite State Machines (FSM)]~\cite[Sec~3.4]{harris2022digital}
\end{definition}
\vspace{5em}

\begin{definition}[Mealy FSM]~\cite[Sec~3.4.3]{harris2022digital}
\end{definition}
\vspace{5em}

\begin{definition}[Moore FSM]~\cite[Sec~3.4.3]{harris2022digital}
\end{definition}
\vspace{5em}


\begin{example}
  A sequential circuit has one input (X) and one output (Z). The circuit
  examines groups of four consecutive inputs and produces an output Z=1 if the
  input sequence 0010 or 0001 occurs. The sequences can overlap. Draw both Mealy
  and Moore timing diagrams. Find the Mealy and Moore state graph.
\end{example}
\vspace{20em}

\begin{prob}
  A sequential circuit has one input (X) and one output (Z). The circuit
  examines groups of four consecutive inputs and produces an output Z=1 if the
  input sequence 0101 or 1001 occurs. The circuit resets after every four
  inputs. Draw both Mealy and Moore timing diagrams. Find the Mealy and Moore state graph.
\end{prob}
\vspace{20em}

\section{Full procedure for designing sequential logic circuit}

\begin{enumerate}
  \item Convert the word problem to a state transition diagram. Let the states
    be $S_0, S_1, S_2, \dots, S_n$.
  \item Draw state transition table with named states. For example,\\
    \begin{tabular}{lllll}
      \toprule
      Present State & \multicolumn{2}{c}{Next State} & \multicolumn{2}{c}{Outputs} \\
      &  X = 0 & X = 1& X=0 & X=1  \\
      \midrule
       $S_0$ & $S_1$ & $S_2$ & 0 & 0\\
      $S_1$ & $S_2$ & $S_0$ & 0 & 0\\
      \vdots & \vdots & \vdots & \vdots & \vdots  \\
      \bottomrule
    \end{tabular}
   \item State reduction step: Reduce the number of required states to a
     minimum. Eliminate unnecessary or duplicate states.
   \item State assignment step: Assign each state a binary representation. For
     example, \\
     \begin{tabular}{ll}
       \toprule
       State name & State assignments ($Q_2 Q_1 Q_0$)\\
       \midrule
       $S_0$ & 000 \\
       $S_1$ & 001 \\
       \vdots & \vdots\\
       \bottomrule
     \end{tabular}
     
   \item Draw State assigned transition table. For example,\\
     \begin{tabular}{llllll}
       \toprule
       \multicolumn{2}{c}{Inputs ($X_1 X_0$) } & Present State ($Q_1 Q_0$) & Next State ($Q_1^+Q_0^+$)& \multicolumn{2}{c}{Outputs ($Z_1 Z_0$)} \\
       \midrule
       0 & 0 & 00 & 01 & 0 & 0\\
       0 & 0 & 01 & 10 & 0 & 0\\
       \vdots & \vdots & \vdots & \vdots & \vdots & \vdots \\
       \bottomrule
     \end{tabular}
     \begin{enumerate}
       \item Use excitation tables to find truth tables for the combinational
         circuits. For example, the excitation table for J-K ff is\\
         \begin{tabular}{ll|ll}
           \toprule
           Q & $Q^{+}$ & J & K\\
           \midrule
           0 & 0 & 0 & d \\
           0 & 1 & 1 & d \\
           1 & 0 & d & 1 \\
           1 & 1 & d & 0 \\
           \bottomrule
         \end{tabular}
     \end{enumerate}
\end{enumerate}

\section{State assignment by guideline method ~\cite[Section~8.2.5]{katz2004contemporary}}
\todo{skip this section}
\subsection{State Maps}

\begin{example}
 Draw a state map for a sequential assignment  of the states\\
  \includegraphics[width=0.2\linewidth]{./media/fig8.27-five-state-FSM.png}
\end{example}


\subsection{Guideline method}
Guideline method states that the following states should be adjacent in the
state map according the following priorities:\\
\includegraphics[width=0.3\linewidth]{./media/fig8.29-assignment-priorities.png}


\begin{example}
  A state transition table is given. Find optimal state assignment by using the
  guideline method.\\
  \includegraphics[width=0.7\linewidth]{./media/fig8.12-state-transition-table-3-bit-sequence-detector.png}%
  \includegraphics[width=0.3\linewidth]{./media/fig8.30-state-diagram-for-3-bit.png}
\end{example}

\begin{example}
  Draw a Mealy FSM for detecting binary string 0110 or 1010. The machine returns
  to the reset state after each and every 4-bit sequence. Draw the state
  transition diagram on your own as practice problem. The state transition
  diagram is given here. Find optimal state assignment by using the guideline method.\\
  \includegraphics[width=0.3\linewidth]{./media/fig8.6-4-bit-string.png}
\end{example}



\section{State reduction by implication chart}

\begin{example}
  Design a Mealy FSM for detecting binary sequence 010 or 0110. The machine
  returns to reset state after each and every 3-bit sequence. For now the state
  transition table is given.
  Reduce the following state transition table \\
  \includegraphics[width=\linewidth]{./media/fig8.8-initial-3-bit-seq-det.png}
\end{example}
\newpage


\subsection{Implication chart Summary}
The algorithms for state reduction using the implication chart method consists
of the following steps
\begin{enumerate}
  \item Construct the implication chart, consisting of one square for each
    possible combination of states taken two at a time.
  \item For each square labeled by states $S_i$ and $S_j$, if the outputs of the
    states differ, mark the square with an $X$; the states are not equivalent.
    Otherwise, they may be equivalent. Within the square write implied pairs of
    equivalent next states for all input combinations.
  \item Systematically advance through the squares of the implication chart. If
    the square labeled by states $S_i, S_j$ contains an implied pair $S_m, S_n$
    and square $S_m, S_n$ is marked with an X, then mark $S_i, S_j$ with an $X$.
    Since $S_m, S_n$ are not equivalent, neither are $S_i, S_j$.
  \item Continue executing Step 3 until no new squares are marked with an X.
  \item For each remaining unmarked square $S_i, S_j$, we can conclude that
    $S_i, S_j$ are equivalent.
\end{enumerate}


