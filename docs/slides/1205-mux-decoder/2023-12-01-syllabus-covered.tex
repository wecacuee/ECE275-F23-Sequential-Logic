\documentclass[options]{article}
\usepackage{enumitem,amssymb}
\newlist{todolist}{itemize}{2}
\setlist[todolist]{label=$\square$}

\usepackage{pifont}
\newcommand{\cmark}{\ding{51}}%
\newcommand{\xmark}{\ding{55}}%
\newcommand{\done}{\rlap{$\square$}{\raisebox{2pt}{\large\hspace{1pt}\cmark}}%
  \hspace{-2.5pt}}
\newcommand{\wontfix}{\rlap{$\square$}{\large\hspace{1pt}\xmark}}


\begin{document}

\section{Study guide}
\subsection{Final}
\begin{todolist}
  \item[\done] Binary numbers, Hexadecimal, Sign-magnitude, One's-complement and
    Two's complement. Conversions between them.
  \item[\done] Generate minterms, maxterms, SOP canonical form and POS
    canonical forms and convert between them\\
  \item[\done]  Understand and use the laws and theorems of Boolean Algebra
  \item[\done]  Perform algebraic simplification using Boolean algebra
  \item[\done]  Derive sum of product and product of sums expressions for a combinational circuit
  \item[\done]  Convert combinational logic to NAND-NAND and NOR-NOR forms
  \item[\done]  Simplification using Quine-McCluskey method
\end{todolist}

\subsection{Midterm 2}
\begin{todolist}
  \item[\done] Simplification using K-maps
  \item[\done] Understand the difference between synchronous and asynchronous inputs
  \item[\done] Derive a state graph or state table from a word description of the problem
\item [\done] Different between and limitations of level-triggered latches and edge-triggered flip-flops.
  \item[\done] Implement a design using JK, SR, D or T flip-flops
  \item[\done] Analyze a sequential circuit and derive a state-table and a state-graph
  \item[\done] Analyse and design both Mealy and Moore sequential circuits with multiple inputs and multiple outputs
  \item[\done] Reduce the number of states in a state table using row reduction and implication tables
  \item[\done] Perform a state assignment using the guideline method
\end{todolist}

\subsection{Final (includes previous topics)}
\begin{todolist}
\item[\done] Design combinational circuits for positive and negative logic
\item[\done] Design Hazard-free two level circuits.
\item[\done] Compute fan out and noise margin of one device driving the same time
\item[\done] Compute noise margin of one device
\item[\done] Describe how tri-state and open-collector outputs are different from totem-pole outputs.
\item[\done] Know the differences and similarities between FPGA, PLA, and ROMs and can use each for logic design
\item Design combinational circuits using multiplexers and decoders
\item Convert between Mealy and Moore designs
\end{todolist}

% \subsection{Labs}
% \begin{todolist}
%   \item[\done] Use computer tools to enter designs graphically and HDL
%   \item Simulate designs using computer tools
%   \item Use computer tools to program gate arrays logic and debug and test
% \end{todolist}

\end{document}
